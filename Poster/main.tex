\documentclass[20pt, a0paper, portrait]{tikzposter}
\title{\textbf{Interferencias en Multicapas: Espejos de Bragg y Cavidades Resonantes}}
\author{Berredo, L. \hspace{1em} Tresguerres, J., \hspace{1em} Vázquez, M., \hspace{1em} Villar, O.}
\institute{Universidad de Oviedo}

\usetheme{Basic}
\usecolorstyle{Spain}
\usetitlestyle{Empty}

\usepackage[utf8]{inputenc}
\usepackage{xcolor}
\usepackage{qrcode}
\usepackage{hyperref}
\usepackage{amsmath}
\usepackage{amsfonts}
\usepackage[spanish]{babel}
\usepackage{graphicx}


\hypersetup{colorlinks=true, linkcolor=black, urlcolor=black}

\begin{document}
\maketitle


\block{Abstract}{

\textbf{English.} In this poster, the authors discuss how to calculate interference on multilayer materials via mathematical formulation (transfer matrix method) and numerical computation (numerical methods and programming). Applications are studied, via Bragg's mirror and Fabry-Pérot resonant cavities. Finally, physical conclusions are derived and further works are mentioned.

\vspace{1em}

\textbf{Español.} En este póster, se discute cómo calcular interferencia en materiales multicapa mediante una formulación matemática (método matrices de transferencia) y cálculo numérico (métodos numéricos y programación). Se estudian aplicaciones mediante el espejo de Bragg y las cavidades resonantes de tipo Fabry-Pérot. Finalmente, se extraen conclusiones físicas y se mencionan avenidas futuras.
}


\begin{columns}
\column{0.5}

\block{Introducción}{
\textbf{Espejos de Bragg.}

Un espejo de Bragg es un sistema multicapa diseñado para maximizar la reflectancia en un intervalo de longitudes de onda centrado en $\lambda_0$. Consta de $m$ pares de capas de índices $n_H>n_L$, con espesores $d_H=\frac{\lambda_0}{4n_H}$ y $d_L=\frac{\lambda_0}{4n_L}$. A nivel teórico, parar incidencia normal se tiene: 
\begin{equation}
    \frac{\Delta \lambda}{\lambda_0}\approx\frac{4}{\pi}\arcsin \left( \frac{n_H-n_L}{n_H+n_L}\right) \text{ y }R_{\max} = \left( \frac{n_i-n_r}{n_i+n_r} \right)^2 \text{, donde } n_r=n_{sub}\left(\frac{n_H}{n_L}\right)^{2m}
\end{equation}

Para $m\geq10$, $R_{\max}\approx1$. Algunos valores típicos de índices de refracción son: $n_H=2.3$ (TiO$_2$), $n_L=1.45$ (SiO$_2$) en el visible y $n_H=3.5$ (GaAs), $n_L=2.9$ (AlAs) en el infrarrojo. Las aplicaciones van desde el diseño de cables de fibra óptica hasta espejos de alta reflectividad en interferómetros y sensores ópticos. 
\vspace{1em}

\textbf{Cavidades resonantes de Fabry-Pérot.}

Dos espejos de Bragg idénticos enfrentados entre sí forman una cavidad resonante tipo Fabry–Pérot. La luz dentro de la cavidad en resonancia se refuerza por interferencia constructiva dando lugar a bandas de longitudes estrechas. La condición de resonancia a incidencia normal es:
\begin{equation}\label{lr}
    \lambda_q=\frac{2n_{cav}L}{q}, \quad q\in\mathbb{N}
\end{equation}
Este sistema se usa para analizar espectros con gran precisión o fabricación de láseres de emisión controlada en longitud de onda.
}


\column{0.5}
\block{Metodología}{
\textbf{Método de las matrices de transferencia.}

Suponiendo polarización TE, se toman dos superficies $j$ y $j+1$ superpuestas. Al igualar los frentes $E$ y $H$ en dichas intercaras, y usando las ecuaciones de Maxwell ($E_j=\frac{\mu_0c}{n_j}H_j$) se obtiene:

\begin{equation}
\begin{pmatrix}
E_j^- \\
E_j^+
\end{pmatrix}
=
\frac{1}{t_{j,j+1}}
\begin{pmatrix}
e^{i(k_{j+1,z}-k_{j,z})d_j} &
r_{j,j+1}e^{-i(k_{j+1,z}+k_{j,z})d_j} \\
r_{j,j+1}e^{i(k_{j+1,z}+k_{j,z})d_j} &
e^{-i(k_{j+1,z}-k_{j,z})d_j}
\end{pmatrix}
\begin{pmatrix}
E_{j+1}^+ \\
E_{j+1}^-
\end{pmatrix}
\end{equation}
\vspace{1em}

Denotando $M_{j,j+1}$ a la \textit{matriz de transferencia} de la j-ésima capa se sigue que 

\begin{equation}
    \begin{pmatrix}
        E_{1}^- \\ E_1^+
    \end{pmatrix}
    =
    M
    \begin{pmatrix}
        E_{N+1}^- \\ E_{N+1}^+
    \end{pmatrix}\text{ , donde } M=\prod_{i=1}^{n}M_{i,i+1}
\end{equation}

De esta expresión, imponiendo ausencia de onda incidente desde el medio $N+1$, se obtiene:
\begin{equation}
        \begin{pmatrix}
        1 \\ r
    \end{pmatrix}
    =
    M
    \begin{pmatrix}
        t \\ 0
    \end{pmatrix},
\end{equation}

de donde se calculan los coeficientes de transmisión y reflexión. La reflectancia $R$ y transmitancia $T$ son el módulo cuadrado de estos valores y son los que se observan experimentalmente.

}

\end{columns}

\block{Resultados}{



% ---- FILA 1: ÁNGULOS ----
\begin{minipage}[t]{0.22\linewidth}
    Your descriptive text goes here. This column acts as the legend or introduction for the three images to the right.
\end{minipage}
\hfill
\begin{minipage}[t]{0.24\linewidth}
    \centering
    \includegraphics[width=\linewidth]{Bragg-angle=0.jpeg}
\end{minipage}
\hfill
\begin{minipage}[t]{0.24\linewidth}
    \centering
    \includegraphics[width=\linewidth]{Bragg-angle=30.jpeg}
\end{minipage}
\hfill
\begin{minipage}[t]{0.24\linewidth}
    \centering
    \includegraphics[width=\linewidth]{Bragg-angle=60.jpeg}
\end{minipage}





% ---- FILA 2: VARIACIÓN DE m ----
\begin{minipage}[t]{0.22\linewidth}
    Your descriptive text goes here. This column acts as the legend or introduction for the three images to the right.
\end{minipage}
\hfill
\begin{minipage}[t]{0.24\linewidth}
    \centering
    \includegraphics[width=\linewidth]{Bragg-m=1.jpeg}
\end{minipage}
\hfill
\begin{minipage}[t]{0.24\linewidth}
    \centering
    \includegraphics[width=\linewidth]{Bragg-m=2.jpeg}
\end{minipage}
\hfill
\begin{minipage}[t]{0.24\linewidth}
    \centering
    \includegraphics[width=\linewidth]{Bragg-m=4.jpeg}
\end{minipage}


% ---- FILA 3: FABRY-PÉROT ----
\begin{minipage}[t]{0.22\linewidth}
    Your descriptive text goes here. This column acts as the legend or introduction for the three images to the right.
\end{minipage}
\hfill
\begin{minipage}[t]{0.24\linewidth}
    \centering
    \includegraphics[width=\linewidth]{Fabry_Perot-m=1.jpeg}
\end{minipage}
\hfill
\begin{minipage}[t]{0.24\linewidth}
    \centering
    \includegraphics[width=\linewidth]{Fabry_Perot-m=2.jpeg}
\end{minipage}
\hfill
\begin{minipage}[t]{0.24\linewidth}
    \centering
    \includegraphics[width=\linewidth]{Fabry_Perot-m=4.jpeg}
\end{minipage}

}



\begin{columns}
\column{0.5}

\block{Discusión}{

\textbf{Espejos de Bragg}

\begin{itemize}\setlength{\itemsep}{2pt}
\item Al aumentar $\theta_i$ disminuye la componente normal de $k_z$, reduciendo la fase acumulada en cada periodo. Para satisfacer la condición de interferencia constructiva el sistema requiere una $\lambda$ menor, produciéndose un desplazamiento al azul.
\item Al incrementar el número de pares $m$, la reflectancia máxima $R$ aumenta rápidamente y el ancho espectral de la banda $\Delta\lambda$ se amplía. Adopta una forma rectangular en torno al máximo.
\end{itemize}

\textbf{Cavidades de Fabry-Pérot}

\begin{itemize}
\item Incluso con pocos pares de capas se obtiene un notable estrechamiento espectral alrededor de las longitudes de resonancia.
\item Las resonancias de orden superior se concentran hacia menores $\lambda$ según \eqref{lr}; ajustando $L$ puede imponerse $\lambda_0=\lambda_1$. El aumento del número de capas introduce resonancias secundarias, ya que cada par puede comportarse como una cavidad débilmente acoplada.
\end{itemize}
}

\column{0.5}

\block{Bibliografía}{
\begin{thebibliography}{9}
        \itemsep=0pt
        
        \bibitem{hecht} 
        E. Hecht, \textit{Optics}, 5th ed. Pearson, 2017, capítulo 9 (Applications of Single and Multilayer Films).

        \bibitem{als-nielsen} 
        J. Als-Nielsen and D. McMorrow, \textit{Elements of Modern X-ray Physics}, 2nd ed. Wiley, 2011, capítulo 3 (Refraction and Reflection).

        \bibitem{yeh_layered_media}
        P. Yeh, \textit{Optical Waves in Layered Media}, 2nd ed., Wiley-Interscience, (2005), capítulo 6 (Optics of Periodic Layered Media) sección 3 (Bragg Reflectors) y capítulo 7 (Some Applications of Isotropic Layered Media) sección 1 (Fabry-Perot Interferometers).

        \bibitem{yariv_yeh_photonics}
        A. Yariv and P. Yeh, \textit{Photonics: Optical Electronics in Modern Communications}, 6th ed., Oxford University Press, (2006), capítulo 4 (Optical Resonators).
\end{thebibliography}}
\end{columns}

\begin{columns}

\column{0.5}
\block{Conclusiones}{
\begin{itemize}
    \item Al crecer $\theta_i$, la banda se desplaza a menores $\lambda$ (corrimiento al azul).
    \item El aumento del número de pares $m$ incrementa la reflectividad máxima y ensancha la banda prohibida.
    \item La posición espectral puede diseñarse ajustando índices y espesores, mostrando el alto control que permiten estas estructuras.
    \item El método de matrices de transferencia es una muy útil para modelizar sistemas multicapa y diseñar dispositivos fotónicos.
    \item Experimentalmente se observarían pérdidas por absorción y un comportamiento menos idealizado.
\end{itemize}
}

\column{0.5}

\block{Código abierto}{
\begin{minipage}{0.65\linewidth}
    
\textbf{Disponible en el QR}

\begin{itemize}
    \item{Disponible bajo licencia EUPL}
    \item{Versiones en Python y MATLAB}
\end{itemize}


\end{minipage}
\begin{minipage}{0.35\linewidth}

\qrcode[hyperlink, height=2in]{https://github.com/LucasBerredo/MultInterference}


\end{minipage}}

\end{columns}


\end{document}
