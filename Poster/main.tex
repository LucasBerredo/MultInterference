\documentclass[25pt, a0paper, portrait]{tikzposter}
\title{\Huge\textbf{Interferencias en multicapas: espejos de Bragg y cavidades resonantes}}
\author{Berredo, L. \hspace{1em} Tresguerres, J., \hspace{1em} Vázquez, M., \hspace{1em} Villar, O.}
\institute{Universidad de Oviedo}

\usetheme{Basic}
\usecolorstyle{Spain}
\usetitlestyle{Empty}


\usepackage{xcolor}
\usepackage{qrcode}
\usepackage{hyperref}
\usepackage{amsmath}
\usepackage{amsfonts}

\hypersetup{colorlinks=true, linkcolor=black, urlcolor=black}

\begin{document}
\maketitle

\block{Abstract}{

\textbf{English.} In this poster, the authors discuss how to calculate interference on multilayer materials via mathematical formulation (transfer matrix method) and numerical computation (numerical methods and programming). Applications are studied, via Bragg's mirror and Fabry-Pérot resonant cavities. Finally, physical conclusions are derived and further works are mentioned.

\vspace{1em}

\textbf{Español.} En este póster, se discute cómo calcular interferencia en materiales multicapa mediante una formulación matemática (método matrices de transferencia) y cálculo numérico (métodos numéricos y programación). Se estudian aplicaciones mediante el espejo de Bragg y las cavidades resonantes de tipo Fabry-Pérot. Finalmente, se extraen conclusiones físicas y se mencionan avenidas futuras.
}


\begin{columns}
\column{0.5}

\block{Introducción}{
\textbf{Espejo de Bragg.}

Un espejo de Bragg es un sistema multicapa diseñado para maximizar la reflectancia en un intervalo de longitudes de onda centrado en $\lambda_0$. Consta de $m$ pares de capas de índices $n_H>n_L$, con espesores $d_H=\frac{\lambda_0}{4n_H}$ y $d_L=\frac{\lambda_0}{4n_L}$. A nivel teórico se tiene: 
\begin{equation}
    \frac{\Delta \lambda}{\lambda_0}\approx\frac{4}{\pi}\arcsin \left( \frac{n_H-n_L}{n_H+n_L}\right) \text{ y }R_{\max} = \left( \frac{n_i-n_r}{n_i+n_r} \right)^2 \text{, donde } n_r=n_{sub}\left(\frac{n_H}{n_L}\right)^{2m}
\end{equation}

Para $m\geq10$ se tiene $R_{\max}\approx1$. Algunos valores típicos de índices de refracción son: $n_H=2.3$ (TiO$_2$), $n_L=1.45$ (SiO$_2$) en el visible y $n_H=3.5$ (GaAs), $n_L=2.9$ (AlAs) en el infrarrojo. Las aplicaciones van desde el diseño de cables de fibra óptica hasta espejos de alta reflectividad en interferómetros y sensores ópticos. 
\vspace{1em}

\textbf{Cavidad resonante Fabry-Pérot.}

Dos espejos de Bragg idénticos enfrentados forman una cavidad resonante tipo Fabry–Pérot. La luz dentro de la cavidad en resonancia se refuerza por interferencia constructiva dando lugar a bandas de longitudes estrechas. La condición de resonancia es:
\begin{equation}
    2n_{cav}L=q\lambda, \quad q\in\mathbb{N}
\end{equation}
Este sistema se usa para seleccionar y medir luz con alta precisión espectral.
}


\column{0.5}
\block{Metodología}{
\textbf{Método de las matrices de transferencia.}

En el caso más sencillo, se suponen dos superficies $j$ y $j+1$ superpuestas. Al igualar los frentes $E$ y $H$ en dichas intercaras, y usando las ecuaciones de Maxwell ($E_j=\frac{\mu_0c}{n_j}H_j$) se obtiene:

\begin{equation}
\begin{pmatrix}
E_j^- \\
E_j^+
\end{pmatrix}
=
\frac{1}{t_{j,j+1}}
\begin{pmatrix}
e^{i(k_{j+1,z}-k_{j,z})d_j} &
r_{j,j+1}e^{-i(k_{j+1,z}+k_{j,z})d_j} \\
r_{j,j+1}e^{i(k_{j+1,z}+k_{j,z})d_j} &
e^{-i(k_{j+1,z}-k_{j,z})d_j}
\end{pmatrix}
\begin{pmatrix}
E_{j+1}^+ \\
E_{j+1}^-
\end{pmatrix}
\end{equation}
\vspace{1em}

Denotando $M_{j,j+1}$ a la \textit{matriz de transferencia} de la j-ésima capa se sigue que 

\begin{equation}
    \begin{pmatrix}
        E_{1}^- \\ E_1^+
    \end{pmatrix}
    =
    M
    \begin{pmatrix}
        E_{N+1}^- \\ E_{N+1}^+
    \end{pmatrix}\text{ , donde } M=\prod_{i=1}^{n}M_{i,i+1}
\end{equation}

De esta expresión se obtiene:
\begin{equation}
        \begin{pmatrix}
        1 \\ r
    \end{pmatrix}
    =
    M
    \begin{pmatrix}
        t \\ 0
    \end{pmatrix}
\end{equation}

De donde se calculan los coeficientes de transmisión y reflexión. La reflectancia $R$ y transmitancia $T$ es el módulo cuadrado de estos valores y es el observado experimentalmente.

}

\end{columns}

\block{Resultados}{
(Máx 200 palabras)
\begin{itemize}
    \item Como afecta el angulo de incidencia los espejos de Bragg
    \item Como afecta el numero de pares de capas en los espejos de Bragg
    \item ver como el ancho de banda se reduce en las cavidades de Fa
\end{itemize}
}

\block{Discusión}{
(Máx 200 palabras)

}

\begin{columns}

\column{0.5}
\block{Conclusiones}{
(Máx 200 palabras)

}

\column{0.5}
\block{Bibliografía}{
\begin{thebibliography}{9}
        \itemsep=0pt
        
        \bibitem{hecht} 
        E. Hecht, \textit{Optics}, 5th ed. Pearson, 2017, capítulo 9 (Applications of Single and Multilayer Films).

        \bibitem{als-nielsen} 
        J. Als-Nielsen and D. McMorrow, \textit{Elements of Modern X-ray Physics}, 2nd ed. Wiley, 2011, capítulo 3 (Refraction and Reflection).

        \bibitem{yeh_layered_media}
        P. Yeh, \textit{Optical Waves in Layered Media}, 2nd ed., Wiley-Interscience, (2005).

    \bibitem{yariv_yeh_photonics}
    A. Yariv and P. Yeh, \textit{Photonics: Optical Electronics in Modern Communications}, 6th ed., Oxford University Press, (2006).
\end{thebibliography}}

\block{Código abierto}{
\begin{minipage}{0.65\linewidth}
    
\textbf{Disponible en el QR}

\begin{itemize}
    \item{Disponible bajo licencia EUPL}
    \item{Versiones en Python y MATLAB}
\end{itemize}


\end{minipage}
\begin{minipage}{0.35\linewidth}

\qrcode[hyperlink, height=5in]{https://github.com/LucasBerredo/MultInterference}


\end{minipage}}

\end{columns}





\end{document}
